\documentclass[10pt, a4paper]{article}
\usepackage{lrec2006}
\usepackage{graphicx}

\title{ClearTK 2.0:\\ Lessons learned developing a machine learning framework for UIMA}

\name{Steven Bethard$^1$, Philip Ogren$^2$, Lee Becker$^3$}

\address{%
$^1$University of Alabama at Birmingham, Birmingham, AL, USA, \texttt{bethard@cis.uab.edu}\\
$^2$Oracle America, Broomfield, CO, USA, \texttt{philip@ogren.info}\\
$^3$???}


\abstract{}


\begin{document}

\maketitleabstract

\section{Introduction}

\cite{ogren-etal:2008:UIMA-LREC}
\cite{ogren-etal:2009:UIMA-GSCL}

\section{Annotators should look like annotators}

CleartkAnnotator is just a JCasAnnotator

Chunking is just a utility object for use in a JCasAnnotator

Features like TF-IDF are in the Annotator, not in the encoder


\section{Pipelines should look like pipelines}

New evaluation's train and test methods

Trainable extractors for TF-IDF etc. (not encoders)


\section{CollectionReaders should be minimal}

URICollectionReader


\section{Modules should group classes by function}
not organized by type system

cleartk-type-system

cleartk-corpus

cleartk-feature


\section{Type-system-agnostic requires interfaces}

Philip's blog post

weaknesses of OpenNLP approach (e.g., assumes pos is an attribute of token)

ClearNLP work


\section{Users need help past the UIMA overhead}

Write the reader and eval, let the student feature-engineer


\section{Discussion}

\bibliographystyle{lrec2006}
\bibliography{cleartk2}

\end{document}

